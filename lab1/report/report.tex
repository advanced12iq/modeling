\documentclass[a4paper, 14pt]{extarticle}

% Поля
%--------------------------------------
\usepackage{minted}
\usepackage{listings}
\usepackage{amsmath, amssymb, graphicx, array, float}
\usepackage[utf8]{inputenc}
\usepackage{geometry}
\usepackage{listings}
\geometry{a4paper,tmargin=2cm,bmargin=2cm,lmargin=3cm,rmargin=1cm}
\lstset{
    extendedchars=true,
    inputencoding=utf8
}
%--------------------------------------

%Russian-specific packages
%--------------------------------------
\usepackage[T2A]{fontenc}
\usepackage[utf8]{inputenc}
\usepackage[english, main=russian]{babel}
%--------------------------------------

\usepackage{textcomp}
\usepackage{indentfirst}
\usepackage{float}

%Graphics
%--------------------------------------
\usepackage{graphicx}
\usepackage{caption}
\graphicspath{ {./images/} }
\usepackage{wrapfig}
%--------------------------------------

\linespread{1.3}

\sloppy
\clubpenalty=10000
\widowpenalty=10000

\usepackage{enumitem}
\usepackage{caption}
\usepackage{hyperref}

\hypersetup {
    unicode=true
}

\DeclareCaptionLabelSeparator*{emdash}{~--- }
\captionsetup[figure]{labelsep=emdash,font=onehalfspacing,position=bottom}

\usepackage{tempora}
\usepackage{minted}

% Листинги
%--------------------------------------
\usepackage{listings}
\lstset{
  basicstyle=\ttfamily\footnotesize,
  breaklines=true,
  captionpos=t,
  frame=single,
  numbers=left,
  numbersep=5pt,
  xleftmargin=25pt,
  xrightmargin=25pt
}
%--------------------------------------

\usepackage{amsthm,amsfonts,amsmath,amssymb,amscd}
\usepackage{mathtools}
\usepackage[perpage]{footmisc}

%--------------------------------------
\begin{document}
%--------------------------------------

%--------------------------------------
%            ТИТУЛЬНЫЙ ЛИСТ (БЕЗ ИЗМЕНЕНИЙ)
%--------------------------------------
\begin{titlepage}
\thispagestyle{empty}
\newpage

\vspace*{-60pt}
\hspace{-65pt}
\begin{minipage}{0.3\textwidth}
\hspace*{-20pt}\centering
\includegraphics[width=\textwidth]{emblem}
\end{minipage}
\begin{minipage}{0.67\textwidth}\small \textbf{
\vspace*{-0.7ex}
\hspace*{-6pt}\centerline{Министерство науки и высшего образования Российской Федерации}
\vspace*{-0.7ex}
\centerline{Федеральное государственное автономное образовательное учреждение }
\vspace*{-0.7ex}
\centerline{высшего образования}
\vspace*{-0.7ex}
\centerline{<<Московский государственный технический университет}
\vspace*{-0.7ex}
\centerline{имени Н.Э. Баумана}
\vspace*{-0.7ex}
\centerline{(национальный исследовательский университет)>>}
\vspace*{-0.7ex}
\centerline{(МГТУ им. Н.Э. Баумана)}}
\end{minipage}

\vspace{-25pt}
\hspace{-35pt}\rule{\textwidth}{2.3pt}

\vspace*{-20.3pt}
\hspace{-35pt}\rule{\textwidth}{0.4pt}

\vspace{1.5ex}
\hspace{-35pt} \noindent \small ФАКУЛЬТЕТ\hspace{80pt} <<Информатика и системы управления>>

\vspace*{-16pt}
\hspace{47pt}\rule{0.83\textwidth}{0.4pt}

\vspace{0.5ex}
\hspace{-35pt} \noindent \small КАФЕДРА\hspace{50pt} <<Теоретическая информатика и компьютерные технологии>>

\vspace*{-16pt}
\hspace{30pt}\rule{0.866\textwidth}{0.4pt}

\vspace{11em}

\begin{center}
\Large {\bf Лабораторная работа № 1 } \\
\large { <<Сравнение модели Галилея и модели Ньютона>>} \\
\large { по курсу <<Моделирование>>}
\end{center}\normalsize

\vspace{8em}

\begin{flushright}
  {Студент группы ИУ9-81Б: Филимонов М. Д. \hspace*{15pt}\\
  \vspace{2ex}
  Преподаватель: Домрачева А. Б.\hspace*{15pt}}
\end{flushright}

\vfill

\begin{center}
\textsl{Москва 2026}
\end{center}
\end{titlepage}

%--------------------------------------
\section{Постановка задачи}
Требуется сравнить две модели движения тела, брошенного под углом к горизонту из точки $(0,0)$ при параметрах
$\alpha=57^\circ$ и $v_0=160\,\text{м/с}$:
\begin{itemize}
    \item модель Галилея без сопротивления воздуха;
    \item модель Ньютона с квадратичным сопротивлением воздуха.
\end{itemize}

Для модели Ньютона используется система ОДУ по переменным $x, y, u, w$,
коэффициент сопротивления
\[
\beta = \frac{cS\rho}{2}, \quad c=0.15,\; S=3,\; \rho=1.225,
\]
радиус шара задан напрямую: $r=0.25\,\text{м}$,
материал шара --- медь ($\rho_{copper}=8960\,\text{кг/м}^3$).

Необходимо получить и сравнить дальность полета, время полета и максимальную высоту,
сформировать таблицу сравнения и график траекторий.

\section{Исследовательский этап}
В модели Галилея сопротивление воздуха отсутствует, поэтому движение описывается аналитически:
\[
x(t)=x_0+v_0\cos\alpha\,t, \quad
y(t)=y_0+v_0\sin\alpha\,t-\frac{gt^2}{2}.
\]
Точка приземления находится из условия $y(t)=0$.

В модели Ньютона учитывается сопротивление воздуха, пропорциональное квадрату скорости,
поэтому используется численное решение системы ОДУ методом Рунге--Кутты 4-го порядка.
Момент приземления уточняется линейной интерполяцией между шагами,
когда координата $y$ меняет знак.

Итоговые значения:
\begin{itemize}
    \item Галилей: $x_{land}=2383.971836\,\text{м}$, $t_{land}=27.357246\,\text{с}$, $y_{max}=917.747757\,\text{м}$;
    \item Ньютон: $x_{land}=1253.623389\,\text{м}$, $t_{land}=22.331246\,\text{с}$, $y_{max}=614.885546\,\text{м}$;
    \item разница дальности: $\Delta x = -1130.348448\,\text{м}$.
\end{itemize}

\begin{figure}[H]
    \centering
    \includegraphics[width=0.9\textwidth]{trajectory_comparison.png}
    \caption{Сравнение траекторий в моделях Галилея и Ньютона}
    \label{fig:trajectory_comparison}
\end{figure}

\section{Конструкторский этап}
Реализация в \texttt{solution.py} декомпозирована на блоки:
\begin{itemize}
    \item структура параметров модели;
    \item аналитический расчет траектории Галилея;
    \item правая часть системы Ньютона и шаг RK4;
    \item численное моделирование до события приземления с интерполяцией;
    \item формирование таблицы сравнения и сохранение CSV;
    \item построение и сохранение графика;
    \item вывод итоговых метрик в консоль.
\end{itemize}

Для воспроизводимости используются фиксированные параметры и
явное сохранение артефактов в файлы \texttt{comparison.csv} и \texttt{trajectory\_comparison.png}.

\section{Технологический этап}
Проектные артефакты лабораторной \texttt{lab1}:
\begin{itemize}
    \item исполняемый скрипт: \texttt{solution.py};
    \item эквивалентный ноутбук: \texttt{solution.ipynb};
    \item результаты выполнения: \texttt{comparison.csv}, \texttt{trajectory\_comparison.png}.
\end{itemize}

Запуск в \texttt{uv}-окружении:
\begin{verbatim}
uv venv
uv pip install numpy matplotlib jupyter
uv run python solution.py
uv run jupyter nbconvert --to notebook --execute solution.ipynb
\end{verbatim}

Полный исходный код проекта приведен ниже:
\begin{listing}[H]
\caption{Полный код проекта из \texttt{Sollution.md}}
\label{lst:sollution}
\inputminted[
    encoding=utf8,
    fontsize=\footnotesize,
    breaklines,
    linenos
]{python}{../Sollution.md}
\end{listing}

\section{Тестирование, измерения и выводы}

\subsection{Проверка соответствия требованиям}
По данным \texttt{validation\_report.md} все обязательные пункты закрыты (PASS):
\begin{itemize}
    \item реализованы обе физические модели (Галилей и Ньютон);
    \item метод RK4 применен для модели Ньютона;
    \item коэффициент $\beta$ вычисляется по требуемой формуле;
    \item учтена масса медного шарика;
    \item точка приземления уточняется интерполяцией;
    \item сформированы \texttt{comparison.csv} и \texttt{trajectory\_comparison.png};
    \item подготовлены \texttt{solution.py}, \texttt{solution.ipynb}, \texttt{explanation.md}, \texttt{simple.md}.
\end{itemize}

\subsection{Результаты тестирования}
Валидация включает базовый запуск и 3 тест-кейса:
\begin{itemize}
    \item базовый запуск \texttt{python solution.py} --- PASS;
    \item тест-кейс 1 (новое условие, $\alpha=57^\circ$) --- PASS;
    \item тест-кейс 2 ($\beta=0$, совпадение с Галилеем в пределах численной ошибки) --- PASS;
    \item тест-кейс 3 ($\alpha=30^\circ$, устойчивость результата) --- PASS.
\end{itemize}

Проверка выполнения в \texttt{uv}-среде также прошла успешно:
\texttt{uv run python solution.py},
\texttt{uv run jupyter nbconvert --to notebook --execute solution.ipynb}.

\subsection{Обнаруженные проблемы}
Критические проблемы не обнаружены.

\subsection{Рекомендации}
\begin{itemize}
    \item при необходимости повысить точность контроля $\beta=0$ можно уменьшить шаг интегрирования;
    \item расширить анализ графиками компонент скорости $u(t)$ и $w(t)$.
\end{itemize}

\subsection{Итоговые выводы}
Реализация лабораторной \texttt{lab1} соответствует обновленным coding-only требованиям:
программа запускается без ошибок, результаты воспроизводимы,
а сравнение моделей выполнено в табличной и графической форме.

\end{document}
