\documentclass[a4paper, 14pt]{extarticle}

% Поля
%--------------------------------------
\usepackage{minted}
\usepackage{listings}
\usepackage{amsmath, amssymb, graphicx, array, float}
\usepackage[utf8]{inputenc}
\usepackage{geometry}
\usepackage{listings}
\usepackage[english, main=russian]{babel}
\geometry{a4paper,tmargin=2cm,bmargin=2cm,lmargin=3cm,rmargin=1cm}
\lstset{
    extendedchars=true,  
    inputencoding=utf8
}
%--------------------------------------

%Russian-specific packages
%--------------------------------------
\usepackage[T2A]{fontenc}
\usepackage[utf8]{inputenc} 
%--------------------------------------

\usepackage{textcomp}
\usepackage{indentfirst}               
\usepackage{float}

%Graphics
%--------------------------------------
\usepackage{graphicx}
\usepackage{caption}
\graphicspath{ {./images/} }
\usepackage{wrapfig}
%--------------------------------------

\linespread{1.3}

\sloppy
\clubpenalty=10000
\widowpenalty=10000

\usepackage{enumitem}
\usepackage{caption} 
\usepackage{hyperref}

\hypersetup {
	unicode=true
}

\DeclareCaptionLabelSeparator*{emdash}{~--- }
\captionsetup[figure]{labelsep=emdash,font=onehalfspacing,position=bottom}

\usepackage{tempora}
\usepackage{minted}

% Листинги
%--------------------------------------
\usepackage{listings}
\lstset{
  basicstyle=\ttfamily\footnotesize,
  breaklines=true,
  captionpos=t,
  frame=single,
  numbers=left,
  numbersep=5pt,
  xleftmargin=25pt,
  xrightmargin=25pt
}
%--------------------------------------

\usepackage{amsthm,amsfonts,amsmath,amssymb,amscd}
\usepackage{mathtools}
\usepackage[perpage]{footmisc}

%--------------------------------------
\begin{document}
%--------------------------------------

%--------------------------------------
%			ТИТУЛЬНЫЙ ЛИСТ (БЕЗ ИЗМЕНЕНИЙ)
%--------------------------------------
\begin{titlepage}
\thispagestyle{empty}
\newpage

\vspace*{-60pt}
\hspace{-65pt}
\begin{minipage}{0.3\textwidth}
\hspace*{-20pt}\centering
\includegraphics[width=\textwidth]{emblem}
\end{minipage}
\begin{minipage}{0.67\textwidth}\small \textbf{
\vspace*{-0.7ex}
\hspace*{-6pt}\centerline{Министерство науки и высшего образования Российской Федерации}
\vspace*{-0.7ex}
\centerline{Федеральное государственное автономное образовательное учреждение }
\vspace*{-0.7ex}
\centerline{высшего образования}
\vspace*{-0.7ex}
\centerline{<<Московский государственный технический университет}
\vspace*{-0.7ex}
\centerline{имени Н.Э. Баумана}
\vspace*{-0.7ex}
\centerline{(национальный исследовательский университет)>>}
\vspace*{-0.7ex}
\centerline{(МГТУ им. Н.Э. Баумана)}}
\end{minipage}

\vspace{-25pt}
\hspace{-35pt}\rule{\textwidth}{2.3pt}

\vspace*{-20.3pt}
\hspace{-35pt}\rule{\textwidth}{0.4pt}

\vspace{1.5ex}
\hspace{-35pt} \noindent \small ФАКУЛЬТЕТ\hspace{80pt} <<Информатика и системы управления>>

\vspace*{-16pt}
\hspace{47pt}\rule{0.83\textwidth}{0.4pt}

\vspace{0.5ex}
\hspace{-35pt} \noindent \small КАФЕДРА\hspace{50pt} <<Теоретическая информатика и компьютерные технологии>>

\vspace*{-16pt}
\hspace{30pt}\rule{0.866\textwidth}{0.4pt}
  
\vspace{11em}

\begin{center}
\Large {\bf Лабораторная работа № 1 } \\ 
\large { <<Сравнение модели Галилея и модели Ньютона>>} \\
\large { по курсу <<Моделирование>>} 
\end{center}\normalsize

\vspace{8em}

\begin{flushright}
  {Студент группы ИУ9-81Б: Филимонов М. Д. \hspace*{15pt}\\ 
  \vspace{2ex}
  Преподаватель: Домрачева А. Б.\hspace*{15pt}}
\end{flushright}

\vfill

\begin{center}
\textsl{Москва 2026}
\end{center}
\end{titlepage}

%--------------------------------------
\section{Цель работы}
% Placeholder
Изучить и сравнить две модели движения тела: аналитическую модель Галилея без сопротивления воздуха и модель Ньютона с квадратичным сопротивлением воздуха, решаемую методом Рунге-Кутты 4-го порядка.

\section{Постановка задачи}
% Placeholder
Для параметров $\alpha=45^\circ$, $v_0=1\,\text{м/с}$ и начальной точки $(0,0)$ требуется построить обе траектории, вычислить точную координату падения по оси $x$, определить время полета и максимальную высоту, а затем сравнить результаты. Для модели Ньютона использовать коэффициент сопротивления $\beta=\frac{cS\rho}{2}$ при $c=0.15$, $S=3$, $\rho=1.225\,\text{кг/м}^3$.

\section{Теоретические основы}

\subsection{Подраздел}
% Placeholder
В модели Галилея сопротивление воздуха отсутствует, поэтому движение по осям описывается формулами
\[
 x(t)=x_0+v_0\cos\alpha\,t,\qquad
 y(t)=y_0+v_0\sin\alpha\,t-\frac{gt^2}{2}.
\]
После исключения времени получаем уравнение траектории
\[
 y(x)=-\frac{g x^2}{2v_0^2\cos^2\alpha}+\tan\alpha\,x+y_0.
\]
Эта модель позволяет аналитически найти дальность полета и время падения.

\subsection{Подраздел}
% Placeholder
В модели Ньютона вводятся компоненты скорости $u=dx/dt$ и $w=dy/dt$. Система уравнений имеет вид
\[
\begin{cases}
\dfrac{dx}{dt}=u, \\
\dfrac{dy}{dt}=w, \\
\dfrac{du}{dt}=-\dfrac{\beta}{m}u\sqrt{u^2+w^2}, \\
\dfrac{dw}{dt}=-g-\dfrac{\beta}{m}w\sqrt{u^2+w^2}.
\end{cases}
\]
Здесь $\beta$ вычисляется по условию задачи, а масса берется для чугунного шарика: $m=\rho_{\text{чугун}}\frac{4}{3}\pi r^3$.

\subsection{Подраздел}
% Placeholder
Для численного решения используется метод Рунге-Кутты 4-го порядка, обеспечивающий хорошую точность на малом шаге интегрирования. Точка падения определяется интерполяцией между соседними шагами, где значение $y$ меняет знак, что позволяет точнее вычислить $x_{\text{land}}$.

\section{Программная реализация}

\begin{minted}[breaklines]{python}
# PLACEHOLDER:
# Вставьте программный код
def newton_rhs(state, params):
    x, y, u, w = state
    g = params["g"]
    beta = params["beta"]
    m = params["m"]
    speed = (u*u + w*w) ** 0.5
    du_dt = -(beta / m) * u * speed
    dw_dt = -g - (beta / m) * w * speed
    return [u, w, du_dt, dw_dt]

def rk4_step(state, dt, rhs, params):
    k1 = rhs(state, params)
    k2 = rhs(state + 0.5 * dt * k1, params)
    k3 = rhs(state + 0.5 * dt * k2, params)
    k4 = rhs(state + dt * k3, params)
    return state + (dt / 6.0) * (k1 + 2*k2 + 2*k3 + k4)
\end{minted}

\section{Анализ результатов}

% Placeholder
По результатам расчета получено:
\begin{itemize}
    \item модель Галилея: $x_{\text{land}}=0.101937\,\text{м}$, $t_{\text{land}}=0.144160\,\text{с}$, $y_{\max}=0.025484\,\text{м}$;
    \item модель Ньютона с сопротивлением: $x_{\text{land}}=0.061637\,\text{м}$, $t_{\text{land}}=0.123878\,\text{с}$, $y_{\max}=0.018945\,\text{м}$;
    \item разница дальности: $\Delta x=x_{\text{newton}}-x_{\text{galileo}}=-0.040300\,\text{м}$.
\end{itemize}
Полученные значения показывают, что учет сопротивления воздуха уменьшает дальность и высоту полета.


\begin{figure}[H]
    \centering
    \includegraphics[width=0.9\textwidth]{trajectory_comparison.png}
    \caption{Сравнение траекторий в моделях Галилея и Ньютона с сопротивлением воздуха}
    \label{fig:result}
\end{figure}

\section{Заключение}

% Placeholder
В работе выполнено сравнение двух моделей движения при одинаковых начальных условиях. Аналитическая модель Галилея дает большую дальность полета, тогда как модель Ньютона с сопротивлением воздуха показывает более реалистичное уменьшение дальности, времени полета и максимальной высоты. Численный метод Рунге-Кутты 4-го порядка корректно решает систему ОДУ и позволяет точно определять точку падения.

\end{document}
